\documentclass[12pt]{article}

\usepackage[utf8]{inputenc}
\usepackage[T1]{fontenc}
\usepackage[russian, english]{babel}
\babeltags{ru = russian}
\usepackage[noglossbreaks]{covington}
\usepackage{booktabs}
% \usepackage{lmodern}
%\babelfont{rm}{DejaVu Serif}
\babelfont{rm}{Gentium Basic}
\babelfont{sf}{DejaVu Sans}

\usepackage[status=draft]{fixme}
\fxsetup{theme=color}

\usepackage[backend=biber,
            bibstyle=biblatex-sp-unified,
            citestyle=sp-authoryear-comp,
            maxcitenames=3,
            maxbibnames=99]{biblatex}
            
\addbibresource{local.bib}

\author{Kirill Maslinsky}

% FIXME: tentative title
% \title{Word embeddings for Bambara: data and applications}
\title{Positional skipgrams for Bambara: a resource for corpus-based studies}

\begin{document}

\maketitle

\begin{abstract}
  This article presents a new online dataset of linguistically rich
  n-gram frequency data for Bambara based on the disambiguated part of
  the Bambara Reference Corpus. The n-grams in the dataset are
  \textit{positional skipgrams} that capture information about
  co-occurrence of lexical items with grammatical categories at
  various relative positions. These n-grams were constructed with the
  aim to leverage those types of information that are available in the
  morphologically annotated corpus of Bambara given the limited amount
  of textual data. The methodology and data used for constructing
  n-grams for Bambara are discussed, followed by brief illustrations
  of how the positional skipgrams data may be employed in corpus-based
  linguistic research.

  Keywords: Bambara, corpus, n-grams, shared data
\end{abstract}

\section{Introduction}

N-grams — fixed-length sequences of adjacent tokens collected from
textual data — have been widely used in computational linguistics and
natural language processing for decades. A frequency list of n-grams
obtained from a corpus has proven to be a simple yet powerful tool to
represent contextual information and sequential phenomena in natural
language.  The idea that is key to the practical success of n-grams in
a wide variety of language modeling tasks (from spelling correction to
speech recognition) is to extract the information encoded in the relative
positioning of linguistic units into a list of easily quantified
atomic “co-occurrence events”.

When used in a general sense, the approach leaves room for flexibility in
choosing how to build n-grams, and what to include in them. Adjacency
constraints can be relaxed to include items occurring anywhere within
a fixed-width context window, thus producing \textit{skipgrams}. There
is also no need to limit the scope to the lexical or graphical level,
as in the traditional word n-grams and letter n-grams, or even to the
surface level in general. In cases when linguistic annotation is
available for the text, it may be used for building n-grams. The most
common example of the latter is to make n-grams from part of speech
tags of subsequent words that reveal recurrent word order patterns.
Thus n-grams can represent phenomena other than plain lexical
co-occurrence.

This article presents a new online dataset of linguistically rich
n-gram frequency data for Bambara based on the disambiguated part of
the Corpus Bambara de R\'ef\'erence\footnote{The corpus search
  interface as well as general info about the corpus are available
  online at: \url{http://cormand.huma-num.fr/}.}
\autocite{vydrin2013bamana}. N-grams in this dataset were constructed
with the aim to capture those types of information that are available
in the morphologically annotated corpus of Bambara.  Beyond the usual
lexical focus, n-grams were supplemented with paradigmatic grammatical
information and positional features that should allow
for inferences to be made about various aspects of
morphosyntax. %based on feature co-occurrence data.

Making this dataset publicly available is a way to provide access to
the linguistic data derived from the full annotated corpus for a wider
audience of students and researchers without disclosing
copyright-protected texts. The data format has to be general enough to
allow open-ended exploration and use of the data in broad areas of
linguistic research, language learning, and downstream NLP tasks. In
my view, the n-grams list format matches this objective
and has the additional benefit of retaining readability by a human as
well as a machine. While simple tabular format makes data easily
quantifiable for research and engineering tasks, for a human reader, a
frequency-ordered n-grams list preserves meaningful linguistic
categories such as lexemes, grammatical tags, and relative word positions in
a sentence. % Which, I argue, make aggregated and comparative data
% directly interpretable as to the morphosyntactic zones of interest

% may direct/steer the search

% structure of the article
The article is structured as follows. Sections
\ref{sec:definition}--\ref{sec:data} explain the methodology and data
used for constructing n-grams for Bambara, followed by section
\ref{sec:applications} with brief illustrations of how the n-gram data
presented here may be employed in corpus-based linguistic research.

\section{Positional Skipgrams}
\label{sec:definition}

The approach used in this article to combine lexical, grammatical, and
positional information in a single n-gram framework is tentatively
labeled here \textit{positional skipgrams}.  To make sense of this
framework, consider a sentence in Bambara that has part of speech tags
defined for each token.

\begin{example}
  \small
  \label{ex:muso}
  \trigloss
  {í       k'      à       dɔ́n     kó      nàta    \textbf{mùso}    tɛ      ná      ɲùman   tóbi    .}
  {pers    pm      pers    v       cop     n       n       pm      n       adj     v       c}
  {{} pm:-5 pers:-4 v:-3 cop:-2 n:-1 \textbf{n:0} pm:1 n:2 adj:3 v:4 c:5}
  {You should know that a greedy woman won't cook a good sauce}
    % 2SG     SBJV    3SG     connaître       QUOT    cupidité        femme IPFV.NEG        sauce   bon     cuire   .}
\end{example}

To generate a list of positional skipgrams out of this sentence words
are taken one by one, and for each word, all part of speech tags that
fall within the fixed-width context window (five words on each side in
our example) are considered a co-occurrence. A numeric index is then
appended to each tag reflecting its relative position to the current
word: 1 indicates the next word to the right, -3 indicates the third
word on the left, and 0 is the word itself. All pairs of a word
with joined positional tags are then recorded in the list of
n-grams. In our example, for the word \textit{mùso} the following
n-grams will be generated: \textit{mùso--n:0}, \textit{mùso--pm:1},
\textit{mùso--n:-1}, etc.  Depending on the task at hand, it may be
convenient to record reverse co-occurrence events (\textit{pm:1--mùso},
etc.) simultaneously to simplify further processing.

Instead of tracking word co-occurrence events, positional skipgrams
record the information on the occurrence of the word in a certain
position in the surface syntactic structure, to the extent that
syntactic information is reflected in the sequence of part of speech
tags. As usual with n-grams, this positional occurrence is represented
as a series of atomic “co-occurrence events”. In this representation,
the structure of the context is lost, but the disparate events (words
and sentences) thus become comparable. For example, two occurrences of
a word can share a significant part of their positional skipgrams
while not sharing that many context words. The same principle makes it
possible to compare different words by the similarity of their
syntactic contexts (in terms of the relative frequencies of their
positional collocates).

% pair of sentences example: 
% with the same key word, different but syntactically similar
% contexts.

% EX2: with different words, similar tag distributions

While the idea of appending the positional index to the collocate is all that
is needed to define positional skipgrams in general, several other
constraints should be observed to make them more relevant as
linguistic data and to make sure that they are tractable in
downstream computational tasks.

\begin{enumerate}
% * do not create lexical positions /sparse data/
\item Note that in the examples above words are never included as
  positional collocates to other words. While technically nothing
  prevents us from doing so, the focus of the method is to relate
  words to the underlying linguistic categories, and more generally,
  to recurrent phenomena at the non-lexical level.  Essentially, what
  we are interested in is the type of contexts that words are likely
  to \textit{share}.  Moreover, in a less-resourced setting where
  lexical data are already sparse, multiplying the lexicon size by the
  positional dimension would be clearly detrimental for statistical
  inference of any kind.
% * do not cross sentence boundaries
\item Since the positional part of speech tags are included as a proxy
  for syntactic structure, it is reasonable to require that n-grams do
  not cross sentence boundaries.  At the same time, punctuation tokens
  could be recorded as collocates to keep track of the relative
  positions of the word in respect to sentence and clause boundaries
  (for instance, the final stop in the example~\ref{ex:muso} that would
  produce \textit{mùso--c:5}).
% * add tag—tag n-grams to keep track of just syntactic regularities
\item To further compensate for lexical sparsity, it makes sense to
  include n-grams consisting of two positional tags alongside
  positional skipgrams with words. For instance, accumulating
  frequency counts for \textit{n:0--pm:1} would help track the
  fact that nouns tend to fill the position before predicative
  markers as an integral part of the data.
\end{enumerate} 


\section{Related work}

% n-grams very traditional (Manning Shutze?)
N-grams are among the earliest and most widely used methods in 
statistical language processing.  % Statistics on n-grams of adjacent
% letters and phomenes proved useful for optical character recognition
% and speech recognition as early as the 1970s\fxerror{Ref: See Willett 1998}. By
% the 1990s, using n-grams of words was a well-established technique in
% language modeling tasks, such as part of speech
% tagging\fxerror{REF: TreeTagger}, or for more linguistically
% oriented tasks like collocation extraction\fxerror{REF: ..Manning 1999}.
The computational and conceptual simplicity of the “default” bi- and
tri-grams of adjacent words favored their usage wherever the
performance of the resulting model was acceptable. Yet it is clear
that related words are not necessarily positioned next to each
other. As computational power and storage capacity grew, the idea of
using word \textit{skipgrams} gained traction as a way to
alleviate the problem of variability in the surface
structures \autocite{guthrie-etal-2006-closer}. \textit{Concgrams}, suggested in
\cite{cheng2006concgrams}, relaxed constraints not only on the distance between
collocates, but also on their relative order, thereby going even further in this
direction. These generalizations of n-grams clearly widen the scope of
syntactic phenomena that n-grams are able to represent, but simultaneously
introduce much more noise in the frequency data.

The “noise” here means unrelated or indirectly related words appearing
together in the n-gram, while n-grams carrying information on
meaningful regularities would constitute a useful “signal”. The
problem of noise is a direct consequence of the simplistic way of
treating context relationships that reduces any syntactic structure to
plain word sequence. Two opposite ways to maintain a decent
signal-to-noise ratio in n-gram data can be attested in the recent
literature. One way is to collect ever more data to let the noisy
co-occurrences be dwarfed by the relevant ones. This became a trend
after the advent of neural networks in language modeling that followed
the success of word2vec \autocite{mikolov2013distributed}. The other approach
is to reduce noise sources in terms of the surface structure by adding
more linguistic structure to the input data. The idea of building
\textit{syntactic n-grams} based on relations in the syntactic tree
rather than the word sequence is characteristic of this latter
position \autocite{sidorov2014syntactic}. The downside of the first method is
that it requires large amounts of textual data to be available for
training. The obvious drawback of the second is that a reliable
syntactic parser is required for it to work. Both of these are
serious, if not blocking, limitations in the context of low-resourced
languages.

The \textit{positional skipgrams} method suggested in this paper sits
somewhere in between the above approaches in terms of balancing signal
and noise in the n-gram data. Contrary to the word2vec approach,
positional skipgrams do require linguistically annotated data for the
input, but the annotation can be rather shallow, like part of speech
tags in the examples above.  By using tags and their relative
positions, the skipgrams are able to capture the signal on syntactic
regularities from the tags, which in their turn aggregate information
from the dictionary and language knowledge added by human annotators. That
is exactly the kind of signal for which sparse lexical data would be
insufficient in the absence of the huge training corpora. At the same
time, given the current state of the art in part of speech tagging, it
seems reasonable to assume that such annotation can be obtained for
significant amounts of text even for lower-resourced
languages. Collecting more annotated data can compensate for the noisy
way of capturing morphosyntactic structures offered by n-grams. But
given the higher frequency and lower diversity of the part of speech tags,
the signal can be expected to overcome noise much sooner compared to
training on raw words.

To summarize, compared to other n-gram building methods positional
skipgrams are characterized by the two distinct features:

\begin{enumerate}
% * positional (explicitly record position of a collocate relative to the
% current word)
\item They explicitly record the position of a collocate relative to
  the current word. Common skipgram-based models may incorporate
  positional information implicitly. In particular, it has been shown
  that word2vec actually benefit from taking distances between words
  into account by using the decreasing weight coefficient for more
  distant words \autocite{levy-etal-2015-improving}. The closest to
  our approach is the work by \textcite{ling-etal-2015-two} that included
  “what words go where” type of information in addition to “what words
  go together” in word2vec by creating separate models for each
  position of a context word relative to the current
  word.
% * cross-level 
\item Positional skipgrams as implemented in this article combine
  features from two different levels of annotation in the form of
  n-grams. This simple cross-level setup seems to be uncommon in
  n-gram applications in recent literature on natural language
  processing. Actually, the n-gram approach adopted in this article
  was motivated by the example \textit{rythmical n-grams} in the work
  of Petr Plech\'a\v{c} in quantitative analysis of poetry, where
  n-grams combining phonemes with their structural position in the
  verse line were used to represent distribution of the phonemes for
  further modeling \autocite[38]{plechac2019}.
\end{enumerate}

%% REFS

% Applications of n‐grams in textual information systems
% Alexander M. Robertson, Peter Willett 1998

\section{Dataset description}
\label{sec:data}

% corbama-net as a corpus
The dataset presented in this article was built using the manually
disambiguated part of the Bambara Reference Corpus (corbama-net). As
of December 2019, the disambiguated subcorpus contains 1.3M words in
1650 documents. The corpus provides token-level morphological
annotation as well as document-level metadata on the author, the
source of the text, and several tags categorizing the medium, genre,
and theme of the text \autocite[on metadata, see for
details:][]{davydov2010towards}. The annotation provided in the corpus
was obtained using the morphological processor Daba based on a
dictionary and a set of rules \autocite{maslinsky2014daba}, followed
by manual disambiguation by Bambara-proficient
operators.

% type of annotation, grammar lists
The annotation layers available in this subcorpus include the
orthographically normalized token (part of the corpus is in the
old Bambara orthography), lemma, part of speech
tag, and a gloss (lexical equivalent) in French. For multi-morpheme
words there is also a recursive structure that annotates each morpheme
with the similar attributes of a form, a part of speech tag, and a
gloss. Grammatical morphemes, as well as standalone function words are
assigned a Leipzig-style formal gloss from a standard list of glosses
for Bambara\footnote{See the full list of the glosses for grammatical
  morphemes and auxiliaries for Bambara at:
  \url{http://cormand.huma-num.fr/gloses.html}.} instead of the French
equivalent.

% normalization: orthography; canonical lemma variants
The main objective of publishing this dataset is to present
quantitative data on morphosyntactic regularities and variation in the
corpus. Hence other types of variation that are attested in the corpus
are not represented, namely orthographic variation, dialectal
variation, and tonal variation. To eliminate this types of variation
only orthographically normalized forms are used throughout the
dataset. All variants of the same lemma (dialectal, tonal, phonetic,
etc.) were transformed to the canonical form, which is
operationalized as the first variant listed for a lexical entry in the
Bamadaba dictionary\footnote{See information on the dictionary at \url{http://cormand.huma-num.fr/bamadaba.html}.}.

% FIXME: find a way to keep glosses in data?
To make the most of the structural information available in the
annotation, the basic positional skipgrams model presented above is
supplemented with the n-grams based on the morpheme-level grammatical
information. To keep data sparsity at a manageable level, the principle
of limiting the right-hand side of the n-grams to the closed-class and
frequent phenomena was observed (see section~\ref{sec:definition} for
details). Thus out of the morpheme-level annotation layer only
morphological tags from a standard list of glosses were taken into
account. The resulting list of skipgrams includes the pairs of the
following form:
\begin{itemize}
\item wordform (or lemma) — part of speech tag + position
\item part of speech tag — part of speech tag + position
\item wordform (or lemma) — standard gloss + position
\item standard gloss — standard gloss + position
\end{itemize}
Numerals and punctuation are not included as the left-hand side items
in the n-grams, but may appear as positional collocates on the
right-hand side.  The context window width for building skipgrams is
defined to be five tokens on each side of the word, but is not allowed
to cross sentence boundaries. Sentence boundaries are included in the
list of positional collocates using a conventional \textit{SENT} tag.

% variants of data: lemmatized; wordforms; morpheme-based(?)
% sequences; tonal/non-tonal
For the convenience of dataset users, the skipgram frequency data
is presented in several variants. First, the data is split
according to the basic lexical item used for building skipgrams that
is either an orthographically normalized wordform, or a canonical
lemma. Second, frequency data on both wordfrom-based and lemma-based
skipgrams are presented in two forms: an aggregated variant showing
total counts for a whole corpus, and a disaggregated variant showing
document-level frequencies. 

% data format
The data is presented in a text-based tabular format. Skipgram
frequency tables are in the TSV (tab separated values) format and
contain the following columns:
\begin{itemize}
\item lexical item, tag or standard gloss;
\item its positional collocate;
\item total frequency of the lexical item/tag/gloss;
\item total frequency of the collocate;
\item frequency of the co-occurrence of the item with the collocate
  (n-gram frequency);
\item a label indicating the type of the collocate (word--tag,
  tag--tag, etc.) to facilitate filtering.
\end{itemize}
The document-level frequency data has an additional column with the
document ID that precedes the list.  Document-level metadata are provided
as a separate CSV file that can be linked to the document-level
skipgram frequency tables based on the value of the document ID
field. 

% FIXME: example of the data

% some basic descriptive statistics
% number of n-grams, lexicon size (words/lemmas)
% hapax legomena share(?)

\section{Possible applications}
\label{sec:applications}

% meant as a demonstration
This section presents a few examples of the ways in which information
contained in the positional skipgrams can be rearranged and explored
to address linguistic queries. The statistical processing of the data
in the examples is intentionally kept to a minimum, in order to
demonstrate conceptual simplicity and interpretability that the lists
of positional skipgrams can offer by themselves.  The examples
presented in this section do not in any sense form an exhaustive list
of the uses for positional skipgrams for linguistics or natural
language processing; they are meant to serve just as illustrations of
possible applications.

\subsection{Lexical comparison}
%% ALT title: 

% gosi—bugo
Lets start with a simple query on lexical semantics where the
application of the positional skipgrams is quite straightforward.  In
Bambara, there is a pair of moderately frequent verbs, \textit{gòsi}
and \textit{bùgɔ}, both of which mean ‘to hit’. Having corpus data at
hand, we may make inferences about the semantic differences of these
two verbs based on the differences in their context distributions.  In
addition to the traditional reading of the concordance for both verbs,
positional skipgrams can offer a summary of morphosyntactic positions of
each verb together with frequency statistics (see table~\ref{tab:bugogosi.freq}). 

\begin{table}[ht]
  \small
  \centering
  \begin{tabular}{llrrr}
    \toprule
    item & collocate & freq1 & freq2 & ngram\\
    \midrule
    bùgɔ\_v & v:0 & 187 & 188431 & 187\\
    bùgɔ\_v & pm:-2 & 187 & 146505 & 126\\
    bùgɔ\_v & pers:-1 & 187 & 179651 & 76\\
    bùgɔ\_v & c:1 & 187 & 130483 & 64\\
    bùgɔ\_v & pers:-3 & 187 & 145917 & 55\\
    bùgɔ\_v & 3SG:-1 & 187 & 81640 & 43\\
    bùgɔ\_v & INF:-2 & 187 & 49982 & 42\\
    bùgɔ\_v & n:-1 & 187 & 309187 & 38\\
    \addlinespace
    gòsi\_v & v:0 & 285 & 188431 & 285\\
    gòsi\_v & pm:-2 & 285 & 146505 & 179\\
    gòsi\_v & n:-1 & 285 & 309187 & 91\\
    gòsi\_v & PFV.TR:-2 & 285 & 21125 & 84\\
    gòsi\_v & num:3 & 285 & 20081 & 75\\
    gòsi\_v & n.prop:-1 & 285 & 41012 & 74\\
    gòsi\_v & conj:2 & 285 & 45496 & 73\\
    gòsi\_v & pers:-1 & 285 & 179651 & 71\\
    \bottomrule
  \end{tabular}
  
  \caption{Top 8 frequent positional skipgrams for \textit{bùgɔ} and
    \textit{gòsi}. Columns indicate: \textit{freq1} — the frequency of
    the verb itself; \textit{freq2} — total frequency of a collocate
    in a corpus; \textit{ngram} — frequency of co-occurrence of a
    collocate with the verb}
  \label{tab:bugogosi.freq}
\end{table}

The data in the table~\ref{tab:bugogosi.freq} essentially presents an
excerpt from the unaltered table of aggregated counts of positional
skipgrams on the whole Bambara corpus. The only operations needed to
get this view are just proper filtering (all lines including
\textit{gòsi\_v} and \textit{bùgɔ\_v}) and sorting (in the descending
order of skipgram frequency). Yet even this simple frequency list
immediately reveals differences in use that point to the semantic
contrast between these two lexical items. While the first two
positions in the list for both verbs are trivial in that they just
reflect the part of speech and the position of the verb in a clause
(\mbox{S AUX O V}), the third position is of particular interest
because it reflects the position of the direct object, and is different
for the two verbs. Taken together, all n-grams that refer to that
position in the top of the lists indicate, that for \textit{bùgɔ},
personal pronouns (especially 3SG) dominate over nouns in the position
of the direct object, while for \textit{gòsi} the position of a direct
object is more equally distributed among nouns, proper nouns, and
personal pronouns. Thus a hypothesis may be formulated that
\textit{bùgɔ} is preferred when talking about hitting people, while
\textit{gòsi} is more general and probably more suitable in talking
about hitting objects.

Interpretation of raw frequency data may be suggestive, but it is
misleading in many cases. While frequencies of the two verbs in
question are on the same order of magnitude, they still differ by a
factor of 1.5, which makes numbers in the two lists not directly
comparable. A more principled way to identify differences in usage
would require some sort of a statistical model that takes into account
the differences in frequencies. There are plenty of approaches to this
task in natural language processing. For the purposes of this
demonstration we adopt a weighted log-odds model suggested in
\textcite{monroe2008}.

To put it simply, the weighted log odds method is used to compare
relative frequencies of two events. For the sake of example, let's
consider the frequency of occurrence of the personal pronoun before
the verbs \textit{bùgɔ} and \textit{gòsi}, respectively. The values of
these frequencies can be found in the rows for \textit{pers:-1}
collocate in the table~\ref{tab:bugogosi.freq}. To decide which verb
personal pronouns co-occur with more often, the overall frequencies of
the verbs should be taken into account. This can be done by
transforming frequencies into odds, that is the ratio of the number of
cases when there is a pronoun in that position to the number of cases
when there is something else. This gives us $76:(187-76)=0.68$ for
bùgɔ, and $71:(285-71)=0.33$ for \textit{gòsi}. By taking the ratio of
these two values (the odds ratio), we immediately find that personal
pronouns are approximately two times more likely to occur before bùgɔ
than before gòsi. It is conventional to take the logarithm of the odds
ratio (log-odds) to make the measurement symmetrical with respect to
the order of values. In the example above, if we were to divide odds
for gòsi by odds for bùgɔ, the result would be close to 1/2. But the
logarithm of 2 is 0.69 while the logarithm of 1/2 is -0.69, which
reflects the fact that the magnitude of the difference is the same in
both cases, and the sign indicates whether the feature in question is
preferred or avoided by the verb that is on top of the ratio.  The
important intuition behind the \textit{weighted} log odds is that for
rare events we may observe the frequencies 2 and 1 that produce the
same $2:1$ ratio, but this observation is much less reliable compared
to the case of observed frequencies of, for instance, 100 and 50. The
magnitude and even the direction of difference in the former case is
more likely to be due to sampling error. Hence the method includes a
correction term in the formula that puts more weight on those
frequency differences that are supported by more evidence
(examples). The values of the weighted log odds for gòsi vs. bùgɔ are
shown in the last column of table~\ref{tab:bugogosi.lo}. Positive
values indicate the prevalence of the collocate with gòsi, and
negative values correspond to higher co-occurrence rate with bùgɔ.

\begin{table}[ht]
  \small
  \centering
  \begin{tabular}{llrrr}
    \toprule
    collocate & ngram\_bùgɔ\_v & ngram\_gòsi\_v & log\_odds\_gòsi\_v\\
    \midrule
    TOP:-1 & 2 & 67 & 3.69\\
    n.prop:-1 & 13 & 74 & 2.87\\
    PFV.TR:-2 & 28 & 84 & 2.03\\
    n:-1 & 38 & 91 & 1.59\\
    v:-2 & 4 & 20 & 1.40\\
    \addlinespace
    prn:-1 & 25 & 9 & -2.16\\
    RECP:-1 & 13 & 2 & -2.00\\
    NOM.F:-1 & 10 & 1 & -1.87\\
    pers:-1 & 76 & 71 & -1.48\\
    PFV.NEG:-2 & 11 & 4 & -1.43\\
    \bottomrule
  \end{tabular}
  
  \caption{Collocates for the two preceding positions for
    \textit{gòsi} and \textit{bùgɔ}, ordered by weighted
    log-odds. Positive log-odds indicate prevalence of a collocate
    with \textit{gòsi}, negative — with \textit{bùgɔ}. Only collocates
    with overall frequency of 10 or more are included in the list}
  \label{tab:bugogosi.lo}
\end{table}

Table~\ref{tab:bugogosi.lo} shows a list of the positional collocates
in the two preceding positions for both verbs, ranked by the magnitude
of the frequency difference as evaluated by weighted
log-odds.\footnote{The computation was performed using the tidylo R
  package \autocite{silge2019tidylo}.} These data support the observation that pronouns
preferentially occur in the position of the direct object with
\textit{bùgɔ}. The list also demonstrates that most of the proper
nouns that fill the position of the direct object for \textit{gòsi}
are toponyms.

The above example demonstrates that positional skipgrams may serve as
a tool to focus the attention and guide the analysis of differences in
lexical usage, though they cannot directly show what the difference
is. In particular, it helps to construct specific hypotheses in terms
of the positional collocates. The tentative hypotheses built using
positional skipgrams may be further explored with a classical
concordance or more sophisticated statistical modeling.

\subsection{Subcorpora comparison}

% the idea of the ngram profile for the authorship classification
% cf. Sidorov ch. 6

Analysis of positional skipgrams need not be limited to the individual
lexical units. The n-gram frequency easily lends itself to aggregation
by any relevant metatextual properties. As a result, it is easy to
obtain a frequency list of positional skipgrams for a subcorpus of
texts that are comparable in some respect. In effect, this method
allows for comparison of positional distributions of lexical items and
grammatical tags across genres, time periods, regions, etc.

The idea that a frequency list of n-grams for a certain corpus
characterizes the language variety used in the texts is not new. In
the literature on natural language processing and on stylometry it is
known as \textit{n-gram profile}. N-gram profiles can be used to
formally distinguish between different language varieties, provided
that corresponding textual corpora are available to collect n-grams.
It was successfully applied, for instance, in tasks to detect
language by script (with character n-grams) \autocite{cavnar1994n}, and
in authorship attribution \autocite{koppel2003exploiting}.

% newspapers vs. epic poetry
In the following example, two subsets of the Bambara corpus are
contrasted using n-gram profiles built from positional skipgrams:
folkloric texts versus news articles. These two broad genres can be
reasonably expected to differ in many respects of language use, some
of which should clearly manifest itself in the prevailing syntactic
patterns as well as in frequency distributions of part of speech tags,
grammatical categories, and lexical items. The point is not to use
positional skipgrams in a statistical classification setting
(predictive modeling), but to employ them as a guide in the search for
linguistically meaningful contrasts in language use.

% general freq results
% a look into log-odds (what it shows us)
The disambiguated part of the Bambara corpus contains 148 files
classified as folklore (0.25M words in total), and 834 files of news
articles (0.36M words). The n-gram profiles for the two subcorpora
were built using the file-level positional-skipgrams data and the
metadata table. Even a quick inspection of the top-frequency
skipgrams lists for the two genres shows an appreciable difference in
the syntactic patterns of the two subcorpora. The folkloric subcorpus has
the first person singular pronoun \textit{à} as the most frequent feature
and the top 10 is dominated by n-grams involving verbs and personal
pronouns. Contrariwise, all top 10 n-grams for news include a noun,
and most of them consist of two nouns in some positional relationship. The
third person singular occurs only on the 13th line. This clearly attests
to the higher frequency of nouns and longer noun groups. When the weighted
log-odds test discussed in the previous section is applied to the
folklore/news dichotomy, it confirms that the syntactic differences in the
narrative and reported speech versus noun groups is the most prominent
contrast (see table~\ref{tab:folknews.lo}).

\begin{table}[ht]
  \centering
  \begin{tabular}{lrrrr}
    \toprule
    skipgram & log\_odds\_folk & log\_odds\_news & f\_folk & f\_news\\
    \midrule
    pers -- pers:-2 & 30.58 & -30.58 & 7046 & 3716\\
    pers -- pm:1 & 30.11 & -30.11 & 10572 & 7216\\
    kó\_cop -- pers:-1 & 28.69 & -28.69 & 2640 & 463\\
    pers -- v:3 & 28.23 & -28.23 & 7694 & 4752\\
    3SG -- QUOT:1 & 27.19 & -27.19 & 2156 & 286\\
    kó\_cop -- 3SG:-1 & 27.18 & -27.18 & 2155 & 286\\
    \midrule
    n -- n:1 & -28.66 & 28.66 & 5435 & 17621\\
    n.prop -- n.prop:1 & -25.79 & 25.79 & 608 & 5200\\
    n -- num:1 & -25.56 & 25.56 & 1780 & 8243\\
    n.prop -- n.prop:2 & -24.17 & 24.17 & 339 & 3978\\
    n -- n:4 & -23.66 & 23.66 & 7075 & 18918\\
    \bottomrule
  \end{tabular}
  
  \caption{Skipgrams most characteristic of folklore and news
    subcorpora, ordered by weighted log-odds}
  \label{tab:folknews.lo}
\end{table}

% a specific inquiry into NMLZ (silter NMLZ)
The same data and method may be used to explore subtler differences
between these subcorpora, and to test more specific hypotheses about
their differences. As an example, nominalized forms can be taken,
since they are expected to be much more prominent in news. To get an
overview of the differences between folklore and news in respect to
nominalizations, it suffices to filter the skipgrams list to get the
lines containing a reference to the nominalization marker (standard
gloss — \textit{NMLZ}). The differences here are not so pronounced,
but they do exist (see table~\ref{tab:nmlz}). 

\begin{table}[ht]
  \centering
  \begin{tabular}{lrrrr}
    \toprule
    skipgram & log\_odds\_folk & log\_odds\_news & f\_folk & f\_news\\
    \midrule
    sɔ̀sɔli\_n -- NMLZ:0 & 2.48 & -2.48 & 19 & 3\\
    dún\_n -- NMLZ:0 & 2.03 & -2.03 & 141 & 139\\
    nà\_v -- NMLZ:1 & 1.73 & -1.73 & 13 & 4\\
    kòlijí\_n -- NMLZ:0 & 1.53 & -1.53 & 10 & 3\\
    IPFV.NEG -- NMLZ:1 & 1.47 & -1.47 & 20 & 12\\
    \midrule
    PL -- NMLZ:1 & -11.83 & 11.83 & 18 & 741\\
    NMLZ -- PP:1 & -7.83 & 7.83 & 36 & 419\\
    yé\_pp -- NMLZ:-1 & -7.83 & 7.83 & 36 & 419\\
    lá\_pp -- NMLZ:-1 & -7.53 & 7.53 & 81 & 526\\
    NMLZ -- ADR:1 & -7.33 & 7.33 & 27 & 353\\
    ni\_conj -- NMLZ:2 & -6.74 & 6.74 & 3 & 230\\
\bottomrule
\end{tabular}

  \caption{Skipgrams that include nominalization, ordered by the
    weighted log-odds difference between folklore and news. Items with
    overall frequency less than 10 are omitted}
  \label{tab:nmlz}
\end{table}

% \subsection{Word embeddings}

% Corpus size is the main limiting factor. Even in the effort to cover
% as many languages as possible (157) Bambara is not
% included. (Wikipedia size)

% % https://meta.wikimedia.org/wiki/List_of_Wikipedias
% % https://bm.wikipedia.org/w/index.php?title=Sp%C3%A9cial:Statistiques&action=raw

% % SIZE 661 content pages (52436 words total)
% no easily accessable data 

% DISCUSSION — ?
% \section{Discussion}
\section{Conclusion}

This article presented a new dataset built using
a morphologically-annotated and manually disambiguated subcorpus of the
Bambara Reference Corpus, and demonstrated several ways in which this
dataset may help in formulating linguistic hypotheses about various
contrasts present in textual data. This quantitative data (with
metadata) enables testing hypotheses and building models based on
lexico-grammatical distributions in various parts of the corpus.  The
goal of publication of this dataset is to provide a wider audience with
access to the data on linguistic regularities observed in the Bambara
corpus for linguistic research and development of NLP
applications.

The data is organized in the form of frequency lists of
\textit{positional skipgrams}, which is a framework for building
n-grams suggested in this article. These n-grams capture information
about co-occurrence of lexical items with grammatical categories at
various relative positions. This framework was introduced instead of
the more traditional approaches to n-gram building in order to
alleviate the problem of data sparsity. Traditional word-level
skip-grams deliberately throw out information encoded in the exact
positioning of words in the data, regarding it as noise. This may
indeed work when there is a large amount of data. But when the data is
relatively scarce, sampling errors with sparse lexical items is a very
serious concern. Traditionally, co-occurrence analysis treats lexicon
as signal and syntax as noise, whereas in this work I suggest to
switch sides. By tracking positional co-occurrences with higher-level
grammatical categories with positional skipgrams, patterns of the
local surface structures emerge that stem from the information
obtained using dictionaries and human language competence. This can
provide a statistical signal that is more reliable and broad than
plain lexical co-occurrence.

% more general appliccability of the approach : if there are
% some pos-tagged data

% extensive testing required to measure performance on various NLP
% tasks (and to localize appropriately the sources of the gains)
The approach suggested is suited well to less-resourced settings
where overall textual data is not easily available but some annotated
texts are present. For linguists, positional skipgrams may serve as an
exploratory tool which, much like the concordance, reorganizes the
textual data in a non-linear fasion in order to reveal
regularities. This is intended not to replace, but to supplement other
views of the corpus, including online search and concordancing.

% As a tool to guide inquiry, a supplement to the standard
% concordancing, frequency-list building and collocation-list building.

\printbibliography{}

\end{document}

