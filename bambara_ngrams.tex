\documentclass[12pt]{article}

\usepackage[utf8]{inputenc}
\usepackage[T1]{fontenc}
\usepackage[russian, english]{babel}
\babeltags{ru = russian}
\usepackage[style=authoryear]{biblatex}
\usepackage[noglossbreaks]{covington}
% \usepackage{lmodern}
%\babelfont{rm}{DejaVu Serif}
\babelfont{rm}{Gentium Basic}
\babelfont{sf}{DejaVu Sans}

\usepackage[status=draft]{fixme}
\fxsetup{theme=color}


\author{Kirill Maslinsky}

% FIXME: tentative title
% \title{Word embeddings for Bambara: data and applications}
\title{Positional skipgrams for Bambara: a resource for corpus-based studies}

\begin{document}

\maketitle

\begin{abstract}
  This paper describes data and experiments ...


  Keywords: Bambara, corpus, n-grams, shared data
\end{abstract}

\section{Introduction}

N-grams — fixed-length sequences of adjacent tokens collected from
textual data — have been widely used in computational linguistics and
natural language processing for decades. A frequency list of n-grams
obtained from a corpus has proven to be a simple yet powerful tool to
represent contextual information and sequential phenomena in natural
language.  The idea that is key to the practical success of n-grams in
a wide variety of language modeling tasks (from spelling correction to
speech recognition) is to extract the information encoded in the relative
positioning of linguistic units into a list of easily quantified
atomic “co-occurrence events”.

When used in a general sense, the approach leaves room for flexibility in
choosing how to build n-grams, and what to include in them. Adjacency
constraints can be relaxed to include items occurring anywhere within
a fixed-width context window, thus producing \textit{skipgrams}. There
is also no need to limit the scope to the lexical or graphical level,
as in the traditional word n-grams and letter n-grams, or even to the
surface level in general. In cases when linguistic annotation is
available for the text, it may be used for building n-grams. The most
common example of the latter is to make n-grams from part of speech
tags of subsequent words that reveal recurrent word order patterns.
Thus n-grams can represent phenomena other than plain lexical
co-occurrence.

This article presents a new online dataset of linguistically rich
n-gram frequency data for Bambara based on the
disambiguated part of the Corpus Bambara de
R\'ef\'erence\footnote{The ccrpus search interface as well as general info
  about the corpus are available online at:
  \url{http://cormand.huma-num.fr/}.}\fxerror{Cite?}. N-grams in this
dataset were constructed with the aim to capture those types of
information that are available in the morphologically annotated corpus
of Bambara.  Beyond the usual lexical focus, n-grams were supplemented
with paradigmatic grammatical information and positional
features\fxwarning{cues?} that should allow for inferences to be made about
various aspects of morphosyntax. %based on feature co-occurrence data.

Making this dataset publicly available is a way to provide access to
the linguistic data derived from the full annotated corpus for a wider
audience of students and researchers without disclosing
copyright-protected texts. The data format has to be general enough to
allow open-ended exploration and use of the data in broad areas of
linguistic research, language learning, and downstream NLP tasks. In
my view, the n-grams list format matches this objective
and has the additional benefit of retaining readability by a human as
well as a machine. While simple tabular format makes data easily
quantifiable for research and engineering tasks, for a human reader, a
frequency-ordered n-grams list preserves meaningful linguistic
categories such as lexemes, grammatical tags, and relative word positions in
a sentence. % Which, I argue, make aggregated and comparative data
% directly interpretable as to the morphosyntactic zones of interest

% may direct/steer the search

% structure of the article
The article is structured as follows. Sections
\ref{sec:definition}--\ref{sec:data} explain the methodology and data
used for constructing n-grams for Bambara, followed by section
\ref{sec:applications} with brief illustrations of how the
n-gram data presented here may be employed in corpus-based linguistic
research, as well as a sample application for a natural language
processing task — building a word-embedding model.  The closing
section presents a discussion of...\fxerror{Bzzzt!}


\section{Positional Skipgrams}
\label{sec:definition}

The approach used in this article to combine lexical, grammatical, and
positional information in a single n-gram framework is tentatively
labeled here \textit{positional skipgrams}.  To make sense of this
framework, consider a sentence in Bambara that has part of speech tags
defined for each token.

\begin{example}
  \label{ex:muso}
  \trigloss
  {í       k'      à       dɔ́n     kó      nàta    \textbf{mùso}    tɛ      ná      ɲùman   tóbi    .}
  {pers    pm      pers    v       cop     n       n       pm      n       adj     v       c}
  {{} pm:-5 pers:-4 v:-3 cop:-2 n:-1 \textbf{n:0} pm:1 n:2 adj:3 v:4 c:5}
  {‘You should know that a greedy woman won't cook a good sauce’}
    % 2SG     SBJV    3SG     connaître       QUOT    cupidité        femme IPFV.NEG        sauce   bon     cuire   .}
\end{example}

To generate a list of positional skipgrams out of this sentence words
are taken one by one, and for each word, all part of speech tags that
fall within the fixed-width context window (five words on each side in
our example) are considered a co-occurrence. A numeric index is then
appended to each tag reflecting its relative position to the current
word: 1 indicates the next word to the right, -3 indicates the third
word on the left, and 0 is the word itself. All pairs of a word
with joined positional tags are then recorded in the list of
n-grams. In our example, for the word \textit{mùso} the following
n-grams will be generated: \textit{mùso--n:0}, \textit{mùso--pm:1},
\textit{mùso--n:-1}, etc.  Depending on the task at hand, it may be
convenient to record reverse co-occurrence events (\textit{pm:1--mùso},
etc.) simultaneously to simplify further processing.

Instead of tracking word co-occurrence events, positional skipgrams
record the information on the occurrence of the word in a certain
position in the surface syntactic structure, to the extent that
syntactic information is reflected in the sequence of part of speech
tags. As usual with n-grams, this positional occurrence is represented
as a series of atomic “co-occurrence events”. In this representation,
the structure of the context is lost, but the disparate events (words
and sentences) thus become comparable. For example, two occurrences of
a word can share a significant part of their positional skipgrams
while not sharing that many context words. The same principle makes it
possible to compare different words by the similarity of their
syntactic contexts (in terms of the relative frequencies of their
positional collocates).

% pair of sentences example: 
% with the same key word, different but syntactically similar
% contexts.

% EX2: with different words, similar tag distributions

While the idea of appending the positional index to the collocate is all that
is needed to define positional skipgrams in general, several other
constraints should be observed to make them more relevant as
linguistic data and to make sure that they are tractable in
downstream computational tasks.\fxwarning{and computationally tractable?}

\begin{enumerate}
% * do not create lexical positions /sparse data/
\item Note that in the examples above words are never included as
  positional collocates to other words. While technically nothing
  prevents us from doing so, the focus of the method is to relate
  words to the underlying linguistic categories, and more generally,
  to recurrent phenomena at the non-lexical level.  Essentially, what
  we are interested in is the type of contexts that words are likely
  to \textit{share}.  Moreover, in a less-resourced setting where
  lexical data are already sparse, multiplying the lexicon size by the
  positional dimension would be clearly detrimental for statistical
  inference of any kind.
% * do not cross sentence boundaries
\item Since the positional part of speech tags are included as a proxy
  for syntactic structure, it is reasonable to require that n-grams do
  not cross sentence boundaries.  At the same time, punctuation tokens
  could be recorded as collocates to keep track of the relative
  positions of the word in respect to sentence and clause boundaries
  (for instance, the final stop in the example~\ref{ex:muso} that would
  produce \textit{mùso--c:5}).
% * add tag—tag n-grams to keep track of just syntactic regularities
\item To further compensate for lexical sparsity, it makes sense to
  include n-grams consisting of two positional tags alongside
  positional skipgrams with words. For instance, accumulating
  frequency counts for \textit{n:0--pm:1} would help track the
  fact that nouns tend to fill the position before the predicative
  markers as an integral part of the data.
\end{enumerate} 


\section{Related work}

% n-grams very traditional (Manning Shutze?)
N-grams are among the earliest and most widely used methods in 
statistical language processing.  Statistics on n-grams of adjacent
letters and phomenes proved useful for optical character recognition
and speech recognition as early as the 1970s\fxerror{Ref: See Willett 1998}. By
the 1990s, using n-grams of words was a well-established technique in
language modeling tasks, such as part of speech
tagging\fxerror{REF: TreeTagger}, or for more linguistically
oriented tasks like collocation extraction\fxerror{REF: ..Manning 1999}.

The computational and conceptual simplicity of the “default” bi- and
tri-grams of adjacent words favored their usage wherever the
performance of the resulting model was acceptable. Yet it is clear
that related words are not necessarily positioned next to each
other. As computational power and storage capacity grew, the idea of
using word \textit{skipgrams} was gained traction as a way to
alleviate the problem of variability in the surface
structures\fxerror{REF: closer2006}. \textit{Concgrams}, suggested in
...\fxerror{REF?}, relaxed constraints not only on the distance between
collocates, but also on their relative order, thereby going even further in this
direction. These generalizations of n-grams clearly widen the scope of
syntactic phenomena that n-grams are able to represent, but simultaneously
introduce much more noise in the frequency data.

The “noise” here means unrelated or indirectly related words appearing
together in the n-gram, while n-grams carrying information on
meaningful regularities would constitute a useful “signal”. The
problem of noise is a direct consequence of the simplistic way of
treating context relationships that reduces any syntactic structure to
plain word sequence. Two opposite ways to maintain a decent
signal-to-noise ratio in n-gram data can be attested in the recent
literature. One way is to collect ever more data to let the noisy
co-occurrences be dwarfed by the relevant ones. This became a trend
after the advent of neural networks in language modeling that followed
the success of word2vec\fxerror{REF: Mikolov 2013}. The other approach
is to reduce noise sources in terms of the surface structure by adding
more linguistic structure to the input data. The idea of building
\textit{syntactic n-grams} based on relations in the syntactic tree
rather than the word sequence is characteristic of this latter
position\fxerror{REF: Sidorov}. The downside of the first method is
that it requires large amounts of textual data to be available for
training. The obvious drawback of the second is that a reliable
syntactic parser is required for it to work. Both of these are
serious, if not blocking, limitations in the context of low-resourced
languages.

The \textit{positional skipgrams} method suggested in this paper sits
somewhere in between the above approaches in terms of balancing signal
and noise in the n-gram data. Contrary to the word2vec approach,
positional skipgrams do require linguistically annotated data for the
input, but the annotation can be rather shallow, like part of speech
tags in the examples above.  By using tags and their relative
positions, the skipgrams are able to capture the signal on syntactic
regularities from the tags, which in their turn aggregate information
from the dictionary and language knowledge added by human annotators. That
is exactly the kind of signal for which sparse lexical data would be
insufficient in the absence of the huge training corpora. At the same
time, given the current state of the art in part of speech tagging, it
seems reasonable to assume that such annotation can be obtained for
significant amounts of text even for lower-resourced
languages. Collecting more annotated data can compensate for the noisy
way of capturing morphosyntactic structures offered by n-grams. But
given the higher frequency and lower diversity of the part of speech tags,
the signal can be expected to overcome noise much sooner compared to
training on raw words.

To summarize, compared to other n-gram building methods positional
skipgrams are characterized by the two distinct features:

\begin{enumerate}
% * positional (explicitly record position of a collocate relative to the
% current word)
\item They explicitly record the position of a collocate relative to
  the current word. Common skipgram-based models may incorporate
  positional information implicitly. In particular, it has been shown
  that word2vec actually benefit from taking distances between words
  into account by using the decreasing weight coefficient for more
  distant words\fxerror{(see GOLDBERG)}. The closest to our approach
  is the work by Ling et al. that included “what words go where” type
  of information in addition to “what words go together” in word2vec
  by creating separate models for each position of a context word
  relative to the current word.\fxerror{REf: two simple}
% * cross-level 
\item Positional skipgrams as implemented in this article combine
  features from two different levels of annotation in the form of
  n-grams. This simple cross-level setup seems to be uncommon in
  n-gram applications in recent literature on natural language
  processing. Actually, the n-gram approach adopted in this article
  was motivated by the example of the work of Petr Plech\'a\v{c} in
  quantitative analysis of poetry, where n-grams combining phonemes with
  their structural position in the verse line were used to represent
  distribution of the phonemes for further modeling.\fxerror{Ref: FIND Plechac!}
\end{enumerate}

%% REFS

% Applications of n‐grams in textual information systems
% Alexander M. Robertson, Peter Willett 1998

\section{Dataset description}
\label{sec:data}

% corbama-net as a corpus
The dataset presented in this article was built using the manually
disambiguated part of the Bambara Reference Corpus
(corbama-net).\fxerror{Ref: BRC} As of December 2019, the
disambiguated subcorpus contains 1.3M words in 1650 documents. The
corpus provides token-level morphological annotation as well as
document-level metadata on the author, the source of the text, and
several tags categorizing the medium, genre, and theme of the
text.\fxerror{See details: Davydov} The annotation provided in the
corpus was obtained using the morphological processor Daba based on a
dictionary and a set of rules\fxerror{Ref: me}, followed by manual
disambiguation by Bambara-proficient operators.\fxwarning{Sounds
  like it was me who annotated it which is not true}

% type of annotation, grammar lists
The annotation layers available in this subcorpus include the
orthographically normalized token (part of the corpus is in the
old\fxwarning{pre-1986?} Bambara orthography), lemma, part of speech
tag, and a gloss (lexical equivalent) in French. For multi-morpheme
words there is also a recursive structure that annotates each morpheme
with the similar attributes of a form, a part of speech tag, and a
gloss. Grammatical morphemes, as well as standalone function words are
assigned a Leipzig-style formal gloss from a standard list of glosses
for Bambara\fxerror{Link: glosslist} instead of the French equivalent.

% normalization: orthography; canonical lemma variants
The main objective of publishing this dataset is to present
quantitative data on morphosyntactic regularities and variation in the
corpus. Hence other types of variation that are attested in the corpus
are not represented, namely orthographic variation, dialectal
variation, and tonal variation. To eliminate this types of variation
only orthographically normalized forms are used throughout the
dataset. All variants of the same lemma (dialectal, tonal, phonetic,
etc.) were transformed to the canonical form, which is
operationalized as the first variant listed for a lexical entry in the
Bamadaba dictionary.\fxerror{Link to Bamadaba}

% FIXME: find a way to keep glosses in data?
To make the most of the structural information available in the
annotation, the basic positional skipgrams model presented above is
supplemented with the n-grams based on the morpheme-level grammatical
information. To keep data sparsity at a manageable level, the principle
of limiting the right-hand side of the n-grams to the closed-class and
frequent phenomena was observed (see section~\ref{sec:definition} for
details). Thus out of the morpheme-level annotation layer only
morphological tags from a standard list of glosses were taken into
account. The resulting list of skipgrams includes the pairs of the
following form:
\begin{itemize}
\item wordform (or lemma) — part of speech tag + position
\item part of speech tag — part of speech tag + position
\item wordform (or lemma) — standard gloss + position
\item standard gloss — standard gloss + position
\end{itemize}
Numerals and punctuation are not included as the left-hand side items
in the n-grams, but may appear as positional collocates on the
right-hand side.  The context window width for building skipgrams is
defined to be five tokens on each side of the word, but is not allowed
to cross sentence boundaries. Sentence boundaries are included in the
list of positional collocates using a conventional \textit{SENT} tag.

% variants of data: lemmatized; wordforms; morpheme-based(?)
% sequences; tonal/non-tonal
For the convenience of dataset users, the skipgram frequency data
is presented in several variants. First, the data is split
according to the basic lexical item used for building skipgrams that
is either an orthographically normalized wordform, or a canonical
lemma. Second, frequency data on both wordfrom-based and lemma-based
skipgrams are presented in two forms: an aggregated variant showing
total counts for a whole corpus, and a disaggregated variant showing
document-level frequencies. 

% data format
The data is presented in a text-based tabular format. Skipgram
frequency tables are in the TSV (tab separated values) format and
contain the following columns:
\begin{itemize}
\item lexical item, tag or standard gloss;
\item its positional collocate;
\item total frequency of the lexical item/tag/gloss;
\item total frequency of the collocate;
\item frequency of the co-occurrence of the item with the collocate
  (n-gram frequency);
\item a label indicating the type of the collocate (word--tag,
  tag--tag, etc.) to facilitate filtering.
\end{itemize}
The document-level frequency data has an additional column with the
document ID that precedes the list.  Document-level metadata are provided
as a separate CSV file that can be linked to the document-level
skipgram frequency tables based on the value of the document ID
field. 

% FIXME: example of the data

% some basic descriptive statistics
% number of n-grams, lexicon size (words/lemmas)
% hapax legomena share(?)

\section{Possible applications}
\label{sec:applications}

% meant as a demonstration

\subsection{Positional distributions}

% gosi—bugo
% В качестве пар слов можно взять, во-1, те же самые bùgɔ и gòsi.

% bùgɔ_v  v:0     187     188431  187
% bùgɔ_v  pm:-2   187     146505  126
% bùgɔ_v  pers:-1 187     179651  76
% bùgɔ_v  c:1     187     130483  64
% bùgɔ_v  pers:-3 187     145917  55
% bùgɔ_v  3SG:-1  187     81640   43
% bùgɔ_v  INF:-2  187     49982   42
% bùgɔ_v  n:-1    187     309187  38

% gòsi_v  v:0     285     188431  285
% gòsi_v  pm:-2   285     146505  179
% gòsi_v  n:-1    285     309187  91
% gòsi_v  PFV.TR:-2       285     21125   84
% gòsi_v  num:3   285     20081   75
% gòsi_v  n.prop:-1       285     41012   74
% gòsi_v  conj:2  285     45496   73
% gòsi_v  pers:-1 285     179651  71

for bugo, personal pronouns dominate over nouns in the previous
position (direct object) 
for gosi, the position of a direct object is much more equally distributed among n, n.prop, and pers

affinity to the perfective construction for gosi, and infinitive
construction for bugɔ


% Ещё хорошая пара: dɔ́gɔman - mìsɛn "маленький" (или дериват последнего mìsɛnman - даже не знаю, что лучше. Или всю троицу?).
% Или вот:
% bèlebele - bònba (можно также  bèlebeleba) "большой".


\subsection{Stylometry}

% the idea of the ngram profile for the authorship classification


% cf. Sidorov ch. 6

% newspapers vs. epic poetry

\subsection{Word embeddings}

Corpus size is the main limiting factor. Even in the effort to cover
as many languages as possible (157) Bambara is not
included. (Wikipedia size)

% https://meta.wikimedia.org/wiki/List_of_Wikipedias
% https://bm.wikipedia.org/w/index.php?title=Sp%C3%A9cial:Statistiques&action=raw

% SIZE 661 content pages (52436 words total)
no easily accessable data 

% DISCUSSION — ?
\section{Discussion}

why bother with positional, when we have traditional?  because of data
sparsity. normal skip-grams deliberately throw out information encoded
in the exact positioning of words in data, regarding it as noise. this
may indeed work so when amount of data is large. but when the data is
relatively scarce, we do not enjoy the effects of central limit
theorem, we much more experience the ...ies of sampling errors with
sparse word data. If traditionally lexicon is seen as signal, and
syntax as noise, in this work I suggest to switch sides. In annotated
data, expectation is that local surface structures are result of
interpretation and a more reliable and broad, than lexis. 

% more general appliccability of the approach : if there are
% some pos-tagged data

% extensive testing required to measure performance on various NLP
% tasks (and to localize appropriately the sources of the gains)

\section{Conclusion}

Goals and uses of the published data: targeted linguistic explorations
and comparisons, quantitative corpus-based research/data, further NLP
applications. 

As a tool to guide inquiry, a supplement to the standard
concordancing, frequency-list building and collocation-list building.

As quantitative data (with metadata) that enables testing hypotheses
and building models based on lexico-grammatical distributions in
various corpus parts. 

As a resource to enable further NLP processing (training models
etc.).

make another way to access and play with corpus statistics in addition
to the existing online search interface

\end{document}

%%% TRASH

Such an inclusive view effectively allows to employ n-grams to

analytical tasks appropriately formulated

provides an aggregated presentation of
meaningful linguistic information (preserving meaningful linguistic
categories — lexemes, grammatical tags, and relative word
positions in a sentence)


was selected as matching this criterion and with
two additional objectives in mind: to be machine-readable, and to
retain readability by a human as well. 

So that if research question can be cast as an appropriate filtering
of n-gram frequencies 


simple tabular format makes it 


that hopefully 

In an easily quantifiable and simultaneously interpretable manner.
To retain readability by a human as well  as a machine. An aggregated
presentation of meaningful linguistic information.

“occurring directly before pm”

To reiterate, (recapitulate), the objectives are ...
* 

% data sparsity problem and a suggested solution
we do not list words on the right-hand side of n-grams. The focus is
on the recurrent phenomena that may include PoS tags, grammatical
morphemes, etc. 


We presume that for each token there is a lemma (normalized wordform),
and a part of speech tag (or we can use a lemma, or a wordform as a
token).

% example sentence

% tokens
% ps
% positional tags
% (translation?)


To capture explicit syntactic structure we may add tag of the word on
top of the 



% keep positional info

% integrate grammatical forms 

% comparable and interpretable lists

% 

that I tentatively call positional skipgrams for the lack of a better
name.

building n-grams along the dimensions of syntax instead
of sequence. The obvious downside of the idea is that we need a
reliable syntactic parser for it to work.

% Syntactic n-grams Sidorov etc. and similar
find a way to compensate for the simplifying (simplistic) way of
treating context relationships (just sequences) that reduces any
syntactic structure just to the plain sequence. a drive to collect
ever more data for language modeling (GPT-2?)

two opposing trends: collect more data, induce less structure
(Mikolov), let the data amount compensate for the noise


 limits the scope of syntactic
 phenomena they are able to represent
 
as even more general
where positional variation is also neglected.  the popularity of the
term rocketed with the advent of deep learning algorithms word2vec.

% skipgram term (made popular by word2vec)
The term skip-gram popularized by Mikolov in 2013, now used in a more
general sense, when counting non directly adjacent words 

Various flavors of n-grams were used
over decades in the literature, making comprehensive survey a hopeless
task. 

Customarily the default setting for collecting n-grams 

The default setting for collecting word n-grams in most applications 

made its way into word-level language modeling no later than 2006. 

later was generalized to include skipgrams (the term can
be traced back at least to 2006). later generalized to words /
language modeling


This trend
emphasizes the power of the model to induce semantic relationships
with a lot of unannotated data.


On the one side, available linguistic annotation is leveraged to 

on one side we incorporate lingustic annnotation 

Rely on the shared context 

which is arguably a way to go with low-resourced languages 

our approach stands somewhere in between, let the positional
collocates keep track of the similarities

% PROBLEM: insensitiveness to word order

% models without word order info suboptimal for tasks involving syntax
% (Two/simple) “what words go where” / “what words go together”
% formulation. Each of the output matrixes is dedicated to predicting
% theoutput for a specific relative position to the center word.

% We present two simple modifications to the models in the popular
% Word2Vec tool, in or- der to generate embeddings more suited to
% tasks involving syntax.


% Plehach — poetry

% pedagogical resource for language learners (phraseology)
